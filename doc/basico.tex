% This file was created automatically from basics.msk.
% DO NOT EDIT!
\Chapter{Basics}

These are basic functions which are not provided by {\GAP} but should.

\>IsBoolean( <O> ) F

Returns 'true' if object <O> is 'true' or 'false' and 'false' otherwise.

\beginexample
gap> IsBoolean( true ); IsBoolean( fail ); IsBoolean ( false );
true
false
true
\endexample


\>DumpObject( <O> ) O

Dumps all information available for object <O>. This information
includes to which categories it belongs as well as its type and 
hashing information used by {\GAP}.

\beginexample
gap> DumpObject( true );
Object( TypeObj := NewType( NewFamily( "BooleanFamily", [ 11 ], [ 11 ] ),
[ 11, 34 ] ), Categories := [ "IS_BOOL" ] )
\endexample


\>DeclareQtfyProperty( <N>, <F> ) F

Declares a quantifiable property named <N> for filter <F>. A
quantifiable property is a property that can be measured according
to some metric. This Declaration actually declares two functions:
a boolean property <N> and an integer property Qtfy<N>.  The
user must provide the method <N>(<O>, <qtfy>) where <qtfy> is
a boolean that tells the method whether to quantify the property or
simply return a boolean stating if the property is 'true' or 'false'. 

\beginexample
gap> DeclareQtfyProperty("Is2Regular",Graphs);
gap> InstallMethod(Is2Regular,"for graphs",true,[Graphs,IsBool],0,
> function(G,qtfy)
>  local m;
>  m:=Length(Filtered(VertexDegrees(G),x->x<>2));
>  if qtfy then
>     return m;
>  else
>     return (m=0);
>  fi;
> end);
\endexample




